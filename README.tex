\documentclass[12pt]{article}
\usepackage[a4paper,margin=2cm]{geometry}
\usepackage[utf8]{inputenc}
\usepackage{float}
\restylefloat{table}
\usepackage{authblk}
\usepackage{graphicx}
\graphicspath{./}
\usepackage{multirow}
\usepackage{adjustbox}
\usepackage{listings}
\lstset{frame=single,breaklines=true,basicstyle=\small}
\title{Linear Cryptanalysis of 2-bit and 4-bit Crypto S-boxes}
\author[1]{Devanjana Chatterjee}
\author[2]{Tania Roy Chaudhury}
\author[3]{Samyadeep Saha}
\affil{Surendranath College, University of Calcutta}
\begin{document}
\begin{abstract}
Cryptography is the method of converting messages in a way that the actual meaning is hidden from an attacker. For converting a plaintext into cipher text, block cipher and stream cipher are used which belongs to symmetric key cryptography. Linear cryptanalysis is an attack on symmetric key block cipher. In this paper, we see linear cryptanalysis on 2-bit and 4-bit crypto S-box. We intend to view a theoretic approach of linear cryptanalysis on 2-bit and 4-bit crypto s-box and thereby analyzing the security of the s-box.

\textbf{Keyword} Cryptography, Linear Cryptanalysis, Block Cipher, S-box.
\end{abstract}
\section{Introduction}
Symmetric key cryptography, Asymmetric key cryptography, Hash function are the three categories of cryptography. Block cipher belongs to symmetric key cryptography. An encryption algorithm that encrypts a block of fixed size at a time is a block cipher. Substitution box (S-Box) have been a part of the block cipher since very long which is used to hide the relationship between the key and the cipher text. Data Encryption Standard (DES) uses the same key for encryption and decryption of data. In this paper, we are working on the 32-DES s-boxes. 

For a 2 bit s box -A 2 bit S-box contains 4 elements from 0 to 3 and the index of each element also varies from 0 to 3. 

For a 4 bit s-box-A 4 bit s box contains 16 elements staring from 0 to F (Hex) and the index of each elements also varies from 0 to F (hex). 

In case of 2 bit s-box. For determining the standard of the 24 s-boxes. We need to count the no. of 2’s in each s-box. If the count of no of 2’s is higher, then we cannot conclude whether the linear equations are present or not. That is, the S- box is good. If the count of no. of 2’s is less, then we can easily conclude linear equations are present or not. 

In case of 4 bit s-box. For determining the standard of the 32 s-boxes, if the number of 8’s in the table are less, it is good for the cryptanalysts because it won’t be safe and are easy to crack. But if the count of number of 8’s in the table are more compared to the other numbers, then the s-box is said to be safe from cryptanalysis.
\section{Linear Cryptanalysis}
Cryptanalysis is a way of getting to know the coded messages without knowing the key. This is done by the hackers by breaking into security systems to have access to encrypted messages. Linear Cryptanalysis is a known plaintext attack where the attacker operates on the linear approximations between plaintext, cipher text and the key.
\subsection{Linear Cryptanalysis on Block Cipher}
In cryptography, S-box is a basic component of symmetric key algorithm which is used to perform substitution. An S-box takes m bits of input and converts it into n bits of output. An m*n s-box can be implemented using a table with $ 2^m $ words of n bits each. 
\subsection{Linear Cryptanalysis on Stream Cipher}
The main aim of cryptography is to hide the data sent by a sender from any attacker so that the data safely reaches only to the receiver. But there are ways in which an attacker can get access to the messages. There are several kinds of attacks by which an attacker can get access to the message. One of them is known plaintext attack in which some of the plaintext and cipher text is known to the attacker. The attacker then somehow figures out the key and then deciphers the message sent by the sender which uses the same key. 

Linear cryptanalysis gives a known plaintext attack on several stream ciphers. It is based on the same principles of linear cryptanalysis on block ciphers introduced by Matsui. 
\subsection{Concept of probability bias}
Probability bias can be defined as- 

No. of actual occurrence/ No. of total occurrence=Probability 

If the probability reaches from $ \frac{1}{2} $ towards $ 1 $, the chances of presence of that linear equation increases.  

If it reaches towards $ 0 $, the chances of absence of that linear equation increases. 

If the probability bias is equal to $ \frac{1}{2} $, it means maximum uncertainty.  

To sum it up- 

If probability bias is equal to $ \frac{1}{2} $ or close to $ \frac{1}{2} $, maximum uncertainty whether the linear equations are present or not. That means, security of that s-box is good and the attacker won’t be able to crack it. The sbox said to be linear crtyptanalysis immune. 

If probability bias reaches towards $ 0 $ or $ 1 $, it is easy to sum up whether the linear equations are present or not. That means, security of that s-box is poor and the attacker is able to crack it.  

Also, probability bias greater than $ \frac{1}{2} $ is better for a crypto s-box. If that is the case then crypto s-box is said to be more linear cryptanalysis immune.
Based on this, we are analysing the security of s-boxes in this paper. 
\section{Linear Cryptanalysis of 2 bit S-boxes}
A 2 bit s box is described in the table \ref{table:1}.
\subsection{2-bit Crypto S-box}
A 2 bit s-box can be written as follows where each element in the first row named as index is the position of the elements of the S-box and the second row named as “s-box” is the elements of the s–box. The values of each element in the first row are unique and sequential in nature since they are the position of elements which is always unique. The values of the second row, which is said to be the elements of the s-box, are unique but are not always sequential.  
\begin{table}[H]
    \centering
    \begin{tabular}{|c|c|c|c|c|c|}
        \hline
        Row & Column & 1 & 2 & 3 & 4 \\
        \hline
        1 & Index & 0 & 1 & 2 & 3 \\
        \hline
        2 & S-box & 0 & 1 & 3 & 2 \\
        \hline
    \end{tabular}
    \caption{Example of 2-bit S-box}
    \label{table:1}
\end{table}
\subsection{Input vectors and Output boolean function of 2-bit crypto s-box}
In the table \ref{table:2}, row 1 is the index value of the s-boxes. Row 2 and Row 3 are taken as the input vectors (IPV1, IPV2). Column 1-4 are the 2-bit binary values of the index (decimal) values. Row 2 is created by taking the first digit of the 2 bit binary value of the index(decimal) value and Row3 is created by taking the second digit of the 2-bit binary value of the index(decimal value). Row 4 is the elements of the taken s-box. Row 5 and row 6 are taken the Output Boolean Functions (OPBF1, OPBF2). Column 1-4 are the 2 bit binary values of the s-box. Row 5, i.e. the OPBF2 is created by taking the first digit of the 2-bit binary value of the s-box. Similarly. Row 6, i.e. the OPBF1 is created by taking the second digit of the 2-bit binary values of the s-box. 
\begin{table}[H]
    \centering
    \begin{tabular}{|c|c|c|c|c|c|}
        \hline
        Row & Column & 1 & 2 & 3 & 4 \\
        \hline
        1 & Index & 0 & 1 & 2 & 3 \\
        \hline
        2 & IPV2 & 0 & 0 & 1 & 1 \\
        \hline
        3 & IPV1 & 0 & 1 & 0 & 1 \\
        \hline
        4 & S-box & 0 & 1 & 3 & 2 \\
        \hline
        5 & OPBF2 & 0 & 0 & 1 & 1 \\
        \hline
        6 & OPBF1 & 0 & 1 & 1 & 0 \\
        \hline
    \end{tabular}
    \caption{IPVs and OPBFs}
    \label{table:2}
\end{table}
\subsection{2-bit linear relations}
In the table \ref{table:3}, the input vectors of the s box are shown under the column Input vectors. There are two input vectors. The output Boolean functions are shown under column of the same name. There are two output BFs. Naming the Input vector (IPV2) as X1 ,  IPV1 as X2 . Naming the output boolean functions (OPBF2) as Y1 , OPBF1 as Y2. 
\begin{table}[H]
    \centering
    \begin{tabular}{|c|c|c|c|c|c|}
        \hline
        Index & \multicolumn{2}{|c|}{Input vectors} & S-box & \multicolumn{2}{|c|}{Output boolean functions} \\
        \hline
        & IPV2 (X1) & IPV1 (X2) & & OPBF2 (Y1) & OPBF1 (Y2) \\
        \hline
        0 & 0 & 0 & 0 & 0 & 0 \\
        \hline
        1 & 0 & 1 & 1 & 0 & 1 \\
        \hline
        2 & 1 & 0 & 3 & 1 & 1 \\
        \hline
        3 & 1 & 1 & 2 & 1 & 0 \\
        \hline
    \end{tabular}
    \caption{Relation between IPVs and OPBFs}
    \label{table:3}
\end{table}
\subsection{Linear Approximation Table of the 2-bit crypto S-box}
The linear approximation table is formed by taking the input vectors along the column of the table and by taking the OPBF’s along the rows of the table. The linear approximation table is formed performing two steps. First is, performing XOR operation between the IPV’s itself. And again performing XOR operation between the OPBF’s itself. After the XOR operation is performed, the task is to check the number of similar bits between the XOR operated terms.  

The linear equations to be checked for similarity are: 

$$ 0=0 $$ 
$$ X1=0 $$
$$ X2=0 $$
$$ X1 \oplus X2=0 $$
$$ Y1=0 $$
$$ Y2=0 $$
$$ Y1 \oplus Y2=0 $$ 
$$ X1=Y1 $$
$$ X2=Y1 $$
$$ X1 \oplus X2=y1 $$
$$ X1=Y2 $$
$$ X2=Y2 $$
$$ X1 \oplus X2=Y2 $$
$$ Y1 \oplus Y2=X1 $$ 
$$ Y1 \oplus y2=x2 $$
$$ Y1 \oplus Y2=X1 \oplus X2 $$

\begin{table}[H]
    \centering
    \begin{tabular}{|c|c|c|c|c|}
        \hline
        & $ 0 $ & $ X1 $ & $ X2 $ & $ X1 \oplus X2 $ \\
        \hline
        $ 0 $ & 4 & 2 & 2 & 2 \\
        \hline
        $ Y1 $ & 2 & 4 & 2 & 2 \\
        \hline
        $ Y2 $ & 2 & 2 & 2 & 4 \\
        \hline
        $ Y1 \oplus Y2 $ & 2 & 2 & 4 & 2 \\
        \hline
    \end{tabular}
    \caption{Linear Approximation Table}
    \label{lat}
\end{table}
The linear approximation table (table \ref{lat}) is formed in the following way- 

Taking the first equation, i.e. $ 0=0 $, we are comparing the number of similar bits between 4bit binary value of 0 and 4 bit binary value of 0. 0000 and 0000 yields 4 similar bits between them. Therefore we place 4 in the first row. Similarly for the second equation $ X1=0 $, we have compared the number of similar bits between X1 (0011 ) and 0 (0000). Let us take the equation $ X1 \oplus X2=Y1 $, In this, first the XOR operation is performed between X1(0011) and X2(0101),and the the result(0110) is compared with Y1(0011) for similarity of bits. All the other equations are checked similarly.  
\subsection{Algebraic Normal Form}
The equation considered for algebraic normal form is – 
\begin{equation}
    \label{eq:1}
    OPBF=a0 \oplus a1 X1 \oplus a2 X2 \oplus a12 X1 X2
\end{equation}
\begin{equation}
    \label{eq:2}
    b0 \oplus b1 Y1 \oplus b2 Y2 \oplus b12 Y1 Y2=a0 \oplus a1 X1 \oplus a2 X2 \oplus a12 X1 X2
\end{equation}
For different values of $ a0, a1, a2, a12, b0, b1, b2, b12 $ equation \ref{eq:2}, we are trying to see if we are able to get the above written 16 linear equations. 
\begin{table}[H]
    \centering
    \begin{tabular}{|c|c|c|c|c|c|c|c|c|}
        \hline
        A0 & A1 & A2 & A12 & B0 & B1 & B2 & B12 & Equation \\
        \hline
        0  & 0  & 0  & 0   & 0  & 0  & 0  & 0   & $ 0=0 $ \\
        \hline
        0  & 1  & 0  & 0   & 0  & 0  & 0  & 0   & $ X1=0 $ \\
        \hline
        0  & 0  & 1  & 0   & 0  & 0  & 0  & 0   & $ X2=0 $ \\
        \hline
        0  & 1  & 1  & 0   & 0  & 0  & 0  & 0   & $ X1 \oplus X2=0 $ \\
        \hline
        0  & 0  & 0  & 0   & 0  & 1  & 0  & 0   & $ Y1=0 $ \\
        \hline
        0  & 1  & 0  & 0   & 0  & 1  & 0  & 0   & $ X1=Y1 $ \\
        \hline
        0  & 0  & 1  & 0   & 0  & 1  & 0  & 0   & $ X2=Y1 $ \\
        \hline
        0  & 1  & 1  & 0   & 0  & 1  & 0  & 0   & $ X1 XOR X2=Y1 $ \\
        \hline
        0  & 0  & 0  & 0   & 0  & 0  & 1  & 0   & $ Y2=0 $ \\
        \hline
        0  & 1  & 0  & 0   & 0  & 0  & 1  & 0   & $ X1=Y2 $ \\
        \hline
        0  & 0  & 1  & 0   & 0  & 0  & 1  & 0   & $ X2=Y2 $ \\
        \hline
        0  & 1  & 1  & 0   & 0  & 0  & 1  & 0   & $ X1 \oplus X2=Y2 $ \\
        \hline
        0  & 0  & 0  & 0   & 0  & 1  & 1  & 0   & $ Y1 \oplus Y2=0 $ \\
        \hline
        0  & 1  & 0  & 0   & 0  & 1  & 1  & 0   & $ Y1 \oplus Y2=X1 $ \\
        \hline
        0  & 0  & 1  & 0   & 0  & 1  & 1  & 0   & $ Y1 \oplus Y2=X2 $ \\
        \hline
        0  & 1  & 1  & 0   & 0  & 1  & 1  & 0   & $ X1 \oplus X2=Y1 \oplus Y2 $ \\
        \hline
    \end{tabular}
    \caption{Equation from algebraic normal form}
    \label{table:5}
\end{table}
What we observe from the table \ref{table:5} is that all the linear equations are obtained for different values of $ a0,a1,a2,b0,b1,b2,b12 $ when put in the equation \ref{eq:2}.
\subsection{Implementation in C}
\lstinputlisting[language=C]{two.c}
\subsection{Output} \label{two}
\lstinputlisting{two.txt}
For the 2 bit crypto s-box ( 0132 )- 

No. of 0’s=0 

No. of 1’s=0 

No. of 2’s=12 

No. of 3’s=0 

No. of 4’s=4 
\subsection{Scrutiny for linear cryptanalysis 2-bit sbox}
In the linear approximation table \ref{lat}  there are 16 cells for 16 2-bit linear equations. The count for similarity of the bits for the binary numbers are put in each cell. 4 in a cell tells that all the bits are similar for that particular relation. 2 in a cell tells that the particular relation is satisfied for 2 2-bit binary conditions and remains unsatisfied for 2-bit binary conditions. This means, there remains the maximum uncertainty for the attacker whether the linear equations exists or not. 
\subsection{Scrutiny for the code of linear cryptanalsis of 2-bit s-box}
From the output of the code executed for the linear cryptanalysis of 2-bit S-box,SECTION \ref{two}, we can see that the no. of 2’s is maximum, i.e. the count for no. of 2’s is higher than the count of other numbers.  

Now, from the concept of probability bias- 

No of 2’s obtained from the code is $ 12 $. 

Therefore $ \frac{12}{16}=0.75 $ which is near to both $ \frac{1}{2} $ and $ 1 $. Therefore, we conclude the s-box is lineacryptanalysis immune.
\subsection{Security conclusion for 24 2-bit s-box}
\begin{table}[H]
    \centering
    \begin{tabular}{|c|c|c|c|c|}
        \hline
        \multicolumn{4}{|c|}{S-box} & No. of 2s \\
        \hline
        0 & 1 & 2 & 3 & 12 \\
        \hline
        0 & 1 & 3 & 2 & 12 \\
        \hline
        0 & 2 & 1 & 3 & 12 \\
        \hline
        0 & 2 & 3 & 1 & 12 \\
        \hline
        0 & 3 & 1 & 2 & 12 \\
        \hline
        1 & 0 & 2 & 3 & 12 \\
        \hline
        1 & 0 & 3 & 2 & 12 \\
        \hline
        1 & 2 & 0 & 3 & 12 \\
        \hline
        1 & 2 & 3 & 0 & 12 \\
        \hline
        1 & 3 & 0 & 2 & 12 \\
        \hline
        1 & 3 & 2 & 0 & 12 \\
        \hline
        2 & 0 & 1 & 3 & 12 \\
        \hline
        2 & 0 & 3 & 1 & 12 \\
        \hline
        2 & 1 & 0 & 3 & 12 \\
        \hline
        2 & 1 & 3 & 0 & 12 \\
        \hline
        2 & 3 & 0 & 1 & 12 \\
        \hline
        2 & 3 & 1 & 0 & 12 \\
        \hline
        3 & 0 & 1 & 2 & 12 \\
        \hline
        3 & 0 & 2 & 1 & 12 \\
        \hline
        3 & 1 & 0 & 2 & 12 \\
        \hline
        3 & 1 & 2 & 0 & 12 \\
        \hline
        3 & 2 & 0 & 1 & 12 \\
        \hline
        3 & 2 & 1 & 0 & 12 \\
        \hline
    \end{tabular}
    \caption{No of 2's in 24 2-bit s-box}
    \label{sec:1}
\end{table}
\section{Linear Cryptanalysis of 4bit S-boxes}
A 4-bit S-box is described in the table \ref{table:6}. 
\subsection{4-bit Crypto S-box}
A 4 bit s-box can be written as follows where each element in the first row named as index is the position of the elements of the S-box and the second row named as “s-box” is the elements of the s –box. The values of each element in the first row are unique and sequential in nature since they are the position of elements which is always unique. The values of the second row, which is said to be the elements of the s box , are unique but are not always sequential. The S-box taken is 0, F, 7, 4, E, 2, D, 1, A, 6, C, B, 9, 5, 3, 8. 
\begin{table}[H]
    \centering
    \begin{tabular}{|c|c|c|c|c|c|c|c|c|c|c|c|c|c|c|c|c|c|}
        \hline
        Row & Column & 1 & 2 & 3 & 4 & 5 & 6 & 7 & 8 & 9 & 10 & 11 & 12 & 13 & 14 & 15 & 16 \\
        \hline
        1 & Index & 0 & 1 & 2 & 3 & 4 & 5 & 6 & 7 & 8 & 9 & A & B & C & D & E & F \\
        \hline
        2 & S-box & 0 & F & 7 & 4 & E & 2 & D & 1 & A & 6 & C & B & 9 & 5 & 3 & 8 \\ 
        \hline
    \end{tabular}
    \caption{4-bit S-box}
    \label{table:6}
\end{table}
\subsection{Input Vectors and Output Boolean Function of 4-bit crypto S-box}
In the table \ref{table:7}, row 1 is the index value of the s-boxes. Row 2 to Row 5 are taken as the input vectors ( IPV1 , IPV2 , IPV3, IPV4 ). Column 1-16 are the 4-bit binary values of the index (decimal) values. Row 2 is created by taking the first digit of the 4 bit binary value of the index(decimal) value ,Row3 is created by taking the second digit of the 4-bit binary value of the index(decimal value) , Row4 is created by taking the third digit of the 4-bit binary value of the index(decimal value), Row5 is created by taking the fourth digit of the 4-bit binary value of the index(decimal value). Row 6 is the elements of the taken s-box. Row 7 and row 10 are taken the Output Boolean Functions ( OPBF1,OPBF2,OPBF3,OPBF4). Column 1-16 are the 4 bit binary values of the s-box. Row 7, i.e. the OPBF4 is created by taking the first digit of the 4-bit binary value of the s-box. Similarly. Row 8, i.e. the OPBF3 is created by taking the second digit of the 4-bit binary value of the s-box, Row 9, i.e. the OPBF2 is created by taking the third digit of the 4-bit binary value of the s-box, Row 10, i.e. the OPBF1 is created by taking the fourth digit of the 4-bit binary value of the s-box. 
\begin{table}[!h]
    \centering
    \begin{tabular}{|c|c|c|c|c|c|c|c|c|c|c|c|c|c|c|c|c|c|}
        \hline
        Row & Column & 1 & 2 & 3 & 4 & 5 & 6 & 7 & 8 & 9 & 10 & 11 & 12 & 13 & 14 & 15 & 16 \\
        \hline
        1 & Index & 0 & 1 & 2 & 3 & 4 & 5 & 6 & 7 & 8 & 9 & A & B & C & D & E & F \\
        \hline
        2 & IPV4(X4) & 0 & 0 & 0 & 0 & 0 & 0 & 0 & 0 & 1 & 1 & 1 & 1 & 1 & 1 & 1 & 1 \\
        \hline
        3 & IPV3(X3) & 0 & 0 & 0 & 0 & 1 & 1 & 1 & 1 & 0 & 0 & 0 & 0 & 1 & 1 & 1 & 1 \\
        \hline
        4 & IPV2(X2) & 0 & 0 & 1 & 1 & 0 & 0 & 1 & 1 & 0 & 0 & 1 & 1 & 0 & 0 & 1 & 1 \\
        \hline
        5 & IPV1(X1) & 0 & 1 & 0 & 1 & 0 & 1 & 0 & 1 & 0 & 1 & 0 & 1 & 0 & 1 & 0 & 1 \\
        \hline
        6 & S-box & 0 & F & 7 & 4 & E & 2 & D & 1 & A & 6 & C & B & 9 & 5 & 3 & 8 \\
        \hline
        7 & OPBF4(Y4) & 0 & 1 & 0 & 0 & 1 & 0 & 1 & 0 & 1 & 0 & 1 & 1 & 1 & 0 & 0 & 1 \\
        \hline
        8 & OPBF3(Y3) & 0 & 1 & 1 & 1 & 1 & 0 & 1 & 0 & 0 & 1 & 1 & 0 & 0 & 1 & 0 & 0 \\
        \hline
        9 & OPBF2(Y2) & 0 & 1 & 1 & 0 & 1 & 1 & 0 & 0 & 1 & 1 & 0 & 1 & 0 & 0 & 1 & 0 \\
        \hline
        10 & OPBF1(Y1) & 0 & 1 & 1 & 0 & 0 & 0 & 1 & 1 & 0 & 0 & 0 & 1 & 1 & 1 & 1 & 0 \\
        \hline
    \end{tabular}
    \caption{4-bit S-box IPVs and OPBFs}
    \label{table:7}
\end{table}
\subsection{4-bit linear relations}
In the table \ref{table:8}, the input vectors of the s box are shown under the column Input vectors. There are four input vectors. The output Boolean functions are shown under column of name output Boolean functions. There are four output BFs . Naming the Input vector (IPV4) as X4 ,  IPV3 as X3 ,IPV2 as X2 and IPV1 as X1. Naming the output boolean functions (OPBF4) as Y4 , OPBF3 as Y3, OPBF2 as Y2 and OPBF1 as Y1. 
\begin{table}[H]
    \centering
    \begin{tabular}{|c|c|c|c|c|c|c|c|c|c|}
        \hline
        \multirow{2}{*}{Index} & \multicolumn{4}{|c|}{Input Vectors} & \multirow{2}{*}{S-box} & \multicolumn{4}{|c|}{Output Boolean Function} \\
        & IPV4 & IPV3 & IPV2 & IPV1 & & OPBF4 & OPBF3 & OPBF2 & OPFB1 \\
        \hline
        0 & 0 & 0 & 0 & 0 & 0 & 0 & 0 & 0 & 0 \\
        \hline
        1 & 0 & 0 & 0 & 1 & F & 1 & 1 & 1 & 1 \\
        \hline
        2 & 0 & 0 & 1 & 0 & 7 & 0 & 1 & 1 & 1 \\
        \hline
        3 & 0 & 0 & 1 & 1 & 4 & 0 & 1 & 0 & 0 \\
        \hline
        4 & 0 & 1 & 0 & 0 & E & 1 & 1 & 1 & 0 \\
        \hline
        5 & 0 & 1 & 0 & 1 & 2 & 0 & 0 & 1 & 0 \\
        \hline
        6 & 0 & 1 & 1 & 0 & D & 1 & 1 & 0 & 1 \\
        \hline
        7 & 0 & 1 & 1 & 1 & 1 & 0 & 0 & 0 & 1 \\
        \hline
        8 & 1 & 0 & 0 & 0 & A & 1 & 0 & 1 & 0 \\
        \hline
        9 & 1 & 0 & 0 & 1 & 6 & 0 & 1 & 1 & 0 \\
        \hline
        A & 1 & 0 & 1 & 0 & C & 1 & 1 & 0 & 0 \\
        \hline
        B & 1 & 0 & 1 & 1 & B & 1 & 0 & 1 & 1 \\
        \hline
        C & 1 & 1 & 0 & 0 & 9 & 1 & 0 & 0 & 1 \\
        \hline
        D & 1 & 1 & 0 & 1 & 5 & 0 & 1 & 0 & 1 \\
        \hline
        E & 1 & 1 & 1 & 0 & 3 & 0 & 0 & 1 & 1 \\
        \hline
        F & 1 & 1 & 1 & 1 & 8 & 1 & 0 & 0 & 0 \\
        \hline
    \end{tabular}
    \caption{Relation between IPVs and OPBFs}
    \label{table:8}
\end{table}
\subsection{Linear Approximation Table (LAT) of the 4-bit crypto s-box}
The linear approximation table is formed by taking the input vectors along the column of the table and by taking the OPBF’s along the rows of the table. The linear approximation table is formed performing two steps. First is, performing XOR operation between the IPV’s itself. And again performing XOR operation between the OPBF’s itself. After the XOR operation is performed, the task is to check the number of similar bits between the XOR operated terms.  
\begin{table}[H]
    \centering
    \begin{adjustbox}{width=\textwidth}
    \begin{tabular}{|c|c|c|c|c|c|c|c|c|c|c|c|c|c|c|c|c|}
        \hline
                    & 0  & X4 & X3 & X2 & X1 & X4,X3 & X4,X2 & X4,X1 & X3,X2 & X3,X1 & X2,X1 & X4,X3,X2 & X4,X3,X1 & X3,X2,X1 & X1,X2,X4 & X4,X3,X2,X1 \\
        \hline
        0           & 16 & 8  & 8  & 8  & 8  & 8     & 8     & 8     & 8     & 8     & 8     & 8        & 8        & 8        & 8        & 8           \\
        \hline
        Y4          & 8  & 10 & 8  & 8  & 6  & 10    & 6     & 8     & 8     & 10    & 6     & 6        & 12       & 10       & 12       & 8           \\
        \hline
        Y3          & 8  & 6  & 6  & 8  & 8  & 8     & 10    & 6     & 10    & 10    & 12    & 8        & 12       & 10       & 6        & 8           \\
        \hline
        Y2          & 8  & 8  & 6  & 6  & 8  & 10    & 6     & 8     & 8     & 10    & 10    & 12       & 6        & 8        & 10       & 12          \\
        \hline
        Y1          & 8  & 8  & 10 & 10 & 8  & 6     & 10    & 8     & 8     & 10    & 10    & 4        & 6        & 8        & 10       & 12          \\
        \hline
        Y4,Y3       & 8  & 12 & 6  & 8  & 10 & 6     & 12    & 6     & 10    & 8     & 6     & 10       & 8        & 8        & 10       & 8           \\
        \hline
        Y4,Y2       & 8  & 10 & 10 & 10 & 6  & 8     & 8     & 8     & 8     & 8     & 12    & 10       & 6        & 10       & 10       & 4           \\
        \hline
        Y4,Y1       & 8  & 10 & 10 & 10 & 6  & 8     & 8     & 8     & 8     & 4     & 8     & 10       & 10       & 10       & 6        & 12          \\
        \hline
        Y3,Y2       & 8  & 10 & 8  & 10 & 8  & 10    & 8     & 10    & 10    & 8     & 10    & 8        & 10       & 2        & 8        & 8           \\
        \hline
        Y3,Y1       & 8  & 10 & 8  & 10 & 8  & 10    & 8     & 10    & 10    & 12    & 6     & 8        & 6        & 10       & 4        & 8           \\
        \hline
        Y2,Y1       & 8  & 8  & 12 & 4  & 8  & 12    & 12    & 8     & 8     & 8     & 8     & 8        & 8        & 8        & 8        & 8           \\
        \hline
        Y4,Y3,Y2    & 8  & 8  & 12 & 6  & 10 & 4     & 6     & 10    & 10    & 10    & 8     & 10       & 10       & 8        & 8        & 8           \\
        \hline
        Y4,Y3,Y1    & 8  & 8  & 8  & 10 & 10 & 8     & 10    & 10    & 2     & 10    & 8     & 10       & 10       & 8        & 8        & 8           \\
        \hline
        Y3,Y2,Y1    & 8  & 6  & 6  & 8  & 8  & 8     & 10    & 14    & 10    & 6     & 8     & 8        & 8        & 10       & 10       & 8           \\
        \hline
        Y1,Y2,Y4    & 8  & 10 & 8  & 8  & 14 & 10    & 6     & 8     & 8     & 6     & 10    & 6        & 8        & 10       & 8        & 8           \\
        \hline
        Y4,Y3,Y2,Y1 & 8  & 4  & 10 & 12 & 10 & 10    & 8     & 6     & 10    & 8     & 6     & 10       & 8        & 8        & 10       & 8           \\
        \hline
    \end{tabular}
    \end{adjustbox}
    \caption{4-bit Linear Approximation Table}
    \label{table:9}
\end{table}
The linear approximation table (table \ref{table:9}) is formed in the following way- 

Taking the first equation, i.e. 0=0, we are comparing the number of similar bits between 16bit binary value of 0 and 16 bit binary value of 0. 0000000000000000 and 0000000000000000 yields 16 similar bits between them. Therefore we place 16 in the first row. Similarly for the second equation X4=0, we have compared the number of similar bits between X4 (0000000011111111 ) and 0 (0000000000000000). Let us take the equation X4 xor X1=Y4, In this, first the XOR operation is performed between X4(0000000011111111) X1(0101010101010101),and the result(0101010110101010) is compared with Y4(0100101010111001) for similarity of bits.No of bits similar found are 8. All the other equations are checked similarly.
\subsection{Implementation in C}
\lstinputlisting[language=C]{four.c}
\subsection{Output} \label{four}
\lstinputlisting{four.txt}
\subsection{Scrutiny for linear cryptanalysis of 4-bit s-box}
In the linear approximation table \ref{table:9}  there are 256 cells for 256 4-bit linear equations. The count for similarity of the bits for the binary numbers are put in each cell. 16 in a cell tells that all the bits are similar for that particular relation. 8 in a cell tells that the particular relation is satisfied for 8 4-bit binary conditions and remains unsatisfied for 8 4-bit binary conditions. This means, there remains the maximum uncertainty for the attacker whether the linear equations exists or not.  
\subsection{Scrutiny for the code of linear crytanalysis of 4-bit s-box}
From the output of the code executed for the linear cryptanalysis of 4-bit S-box,SECTION \ref{four}, we can see that the no. of 8’s is maximum, i.e. the count for no. of 8’s is less than the count of other numbers.  

Now, from the concept of probability bias- 

No of 8’s obtained from the code is $ 119 $. 

Therefore $ \frac{119}{256}=0.46 $ which is near to $ \frac{1}{2} $, but no. of 8's less than 128 this s-box is not linear cryptanalysis immune.
\subsection{Security conclusion for 32 4-bit des s-box}
\begin{table}[H]
    \centering
    \begin{tabular}{|c|c|c|c|c|c|c|c|c|c|c|c|c|c|c|c|c|}
        \hline
        \multicolumn{16}{|c|}{S-box} & No. of 8s \\
        \hline
        14 & 4 & 13 & 1 & 2 & 15 & 11 & 8 & 3 & 10 & 6 & 12 & 5 & 9 & 0 & 7 & 119 \\
        \hline
        0 & 15 & 7 & 4 & 14 & 2 & 13 & 1 & 10 & 6 & 12 & 11 & 9 & 5 & 3 & 8 & 119 \\
        \hline
        4 & 1 & 14 & 8 & 13 & 6 & 2 & 11 & 15 & 12 & 9 & 7 & 3 & 10 & 5 & 0 & 119 \\
        \hline
        15 & 12 & 8 & 2 & 4 & 9 & 1 & 7 & 5 & 11 & 3 & 14 & 10 & 0 & 6 & 13 & 127 \\
        \hline
        15 & 1 & 8 & 14 & 6 & 11 & 3 & 4 & 9 & 7 & 2 & 13 & 12 & 0 & 5 & 10 & 115 \\
        \hline
        3 & 13 & 4 & 7 & 15 & 2 & 8 & 14 & 12 & 0 & 1 & 10 & 6 & 9 & 11 & 5 & 127 \\
        \hline
        0 & 14 & 7 & 11 & 10 & 4 & 13 & 1 & 5 & 8 & 12 & 6 & 9 & 3 & 2 & 15 & 127 \\
        \hline
        13 & 8 & 10 & 1 & 3 & 15 & 4 & 2 & 11 & 6 & 7 & 12 & 0 & 5 & 14 & 9 & 113 \\
        \hline
        10 & 0 & 9 & 14 & 6 & 3 & 15 & 5 & 1 & 13 & 12 & 7 & 11 & 4 & 2 & 8 & 117 \\
        \hline
        13 & 7 & 0 & 9 & 3 & 4 & 6 & 10 & 2 & 8 & 5 & 14 & 12 & 11 & 15 & 1 & 119 \\
        \hline
        13 & 6 & 4 & 9 & 8 & 15 & 3 & 0 & 11 & 1 & 2 & 12 & 5 & 10 & 14 & 7 & 127 \\
        \hline
        1 & 10 & 13 & 0 & 6 & 9 & 8 & 7 & 4 & 15 & 14 & 3 & 11 & 5 & 2 & 13 & 131 \\
        \hline
        7 & 13 & 14 & 3 & 0 & 6 & 9 & 10 & 1 & 2 & 8 & 5 & 11 & 12 & 4 & 15 & 113 \\
        \hline
        13 & 8 & 11 & 5 & 6 & 15 & 0 & 3 & 4 & 7 & 2 & 12 & 1 & 10 & 14 & 9 & 113 \\
        \hline
        10 & 6 & 9 & 0 & 12 & 11 & 7 & 13 & 15 & 1 & 3 & 14 & 5 & 2 & 8 & 4 & 113 \\
        \hline
        3 & 15 & 0 & 6 & 10 & 1 & 13 & 8 & 9 & 4 & 5 & 11 & 12 & 7 & 2 & 14 & 113 \\
        \hline
        2 & 12 & 4 & 1 & 7 & 10 & 11 & 6 & 8 & 5 & 3 & 15 & 13 & 0 & 14 & 9 & 117 \\
        \hline
        14 & 11 & 2 & 12 & 4 & 7 & 13 & 1 & 5 & 0 & 15 & 10 & 3 & 9 & 8 & 6 & 117 \\
        \hline
        4 & 2 & 1 & 11 & 10 & 13 & 7 & 8 & 15 & 9 & 12 & 5 & 6 & 3 & 0 & 14 & 109 \\
        \hline
        11 & 8 & 12 & 7 & 1 & 14 & 2 & 13 & 6 & 15 & 0 & 9 & 10 & 4 & 5 & 3 & 117 \\
        \hline
        12 & 1 & 10 & 15 & 9 & 2 & 6 & 8 & 0 & 13 & 3 & 4 & 14 & 7 & 5 & 11 & 113 \\
        \hline
        10 & 15 & 4 & 2 & 7 & 12 & 9 & 5 & 6 & 1 & 13 & 14 & 0 & 11 & 3 & 8 & 111 \\
        \hline
        9 & 14 & 15 & 5 & 2 & 8 & 12 & 3 & 7 & 0 & 4 & 10 & 1 & 13 & 11 & 6 & 113 \\
        \hline
        4 & 3 & 2 & 12 & 9 & 5 & 15 & 10 & 11 & 14 & 1 & 7 & 6 & 0 & 8 & 13 & 115 \\
        \hline
        4 & 11 & 2 & 14 & 15 & 0 & 8 & 13 & 3 & 12 & 9 & 7 & 5 & 10 & 6 & 1 & 119 \\
        \hline
        13 & 0 & 11 & 7 & 4 & 9 & 1 & 10 & 14 & 3 & 5 & 12 & 2 & 15 & 8 & 6 & 127 \\
        \hline
        1 & 4 & 11 & 13 & 12 & 3 & 7 & 14 & 10 & 15 & 6 & 8 & 0 & 5 & 9 & 2 & 113 \\
        \hline
        6 & 11 & 13 & 8 & 1 & 4 & 10 & 7 & 9 & 5 & 0 & 15 & 14 & 2 & 3 & 12 & 121 \\
        \hline
        13 & 2 & 8 & 4 & 6 & 15 & 11 & 1 & 10 & 9 & 3 & 14 & 5 & 0 & 12 & 7 & 125 \\
        \hline
        1 & 15 & 13 & 8 & 10 & 3 & 7 & 4 & 12 & 5 & 6 & 11 & 0 & 14 & 9 & 2 & 121 \\
        \hline
        7 & 11 & 4 & 1 & 9 & 12 & 14 & 2 & 0 & 6 & 10 & 13 & 15 & 3 & 5 & 8 & 117 \\
        \hline
        2 & 1 & 14 & 7 & 4 & 10 & 8 & 13 & 15 & 12 & 9 & 0 & 3 & 5 & 6 & 11 & 119 \\
        \hline
    \end{tabular}
    \caption{No of 4's in 32 4-bit s-box}
    \label{sec:2}
\end{table}
From the table \ref{sec:2} we can see that none of them are linear cryptanalysis immune, except one which has no of 8's 131. This is a problem in DES s-box.
\section{Conclusion}
In this paper, we saw a detailed theoretical as well as an implementation of that approach for concluding the security of 24 2-bit s box and 32 4-bit DES s-box. We used C program to analyze the security criteria based on the concept of probability bias and with the help of linear approximation table. It can be concluded that for the all the 24 2-bit substitution box ,the security of the s box is good since the probability bias based on the no. of 2's obtained from the code as well as from the linear approximation table is near to 1/2. It can also be concluded that for all the 32 4-bit DES substitution box, the security of the s box except one ,for linear cryptanlaysis is poor since the probability bias based on the no. of 8's obtained from the code as well as the linear approximation table is near to 1/2 but less than 128 . That is the majority of the 32 4-bit DES crypto sbox are not linear cryptanlaysis immune.
\newpage
\textbf{\LARGE References}

1. A Tutorial on Linear and Differential Cryptanalysis by Howard M. Heys 

2. A Review of Existing 4-bit Crypto S-box cryptanalysis Techniques and Two New Techniques with 4-bit Boolean Functions for Cryptanalysis of 4-bit Crypto S-boxes by Sankhanil Dey and Ranjan Ghosh.

3. Edward Schaefer (1996),A Simplified Data Encryption Standard Algorithm, Cryptologia 96. 

4. Susan Landau (2000) Standing the Test of Time: The Data Encryption Standard, Sun Microsystems.

5. H.M.Heys. A tutorial on linear and differential cryptanlysis.cryptologia,26(2002),189-221.

6. H.M.Heys and S.E. Tavares.Substitution-permutation networks resistant to differential and linear cryptanalysis.Journal 
of Cryptology,9(1996),1-19.

7. Mitsuru Matsui(1994). Linear Cryptanalysis method for DES cipher, EUROCRYPT 1994,no.765,pp.386-397.

8. Mitsuru Matsui(1994). The First Experimental Cryptanalysis of Data Encryption Standard.Advances in Cryptology--
CRYPTO'94.1-11

9.Bagheri N. (2015) Linear Cryptanalysis of Reduced-Round SIMECK Variants. In: Biryukov A., Goyal V. (eds) Progress 
in Cryptology -- INDOCRYPT 2015. Lecture Notes in Computer Science,
vol 9462. Springer, Cham.
\end{document}
